\chapter{Estrutura e operações}
\label{chapter: Estrutura e operações}







\begin{commentB}
	
	\par \end{commentB}







\begin{commentA}
Identificação e priorização dos recursos mais importantes requeridos (físicos, financeiros, intelectuais e humanos): quantidades e valores.
\par \end{commentA}



\begin{commentA}
Identificação e priorização das atividades-chave (fluxogramas, funções etc.): produção, resolução de problemas, plataforma/rede.
\par \end{commentA}



\begin{commentA}
Identificação da rede de fornecedores e parceiros principais: pontos fortes e fracos, recursos adquiridos, atividades-chave executadas.
\par \end{commentA}



\begin{commentA}
Quantificação dos custos mais importantes, recursos principais mais caros e atividades-chave mais caras.    
\par \end{commentA}



\begin{commentA}

\par \end{commentA}




