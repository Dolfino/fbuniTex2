\section{\textbf{Apresentação do empreendimento}}
\label{sec: empreendimento}

\begin{commentA}
	
	\par \end{commentA}

\begin{commentB}
	
	\par \end{commentB}

%\subsection{Necessidade de pessoal}

\noindent Dados do empreendimento:\par
\noindent Nome fantasia: T'ENTREGAS.\par
\noindent Endereço: Av. das Dunas, 155, Cumbuco, Caucaia/CE\par
\noindent Telefone: (85) 0000-0000.\par
\noindent CEP: 61619-010.\par
\noindent CNPJ: 00.000.000/0000-00 \par

A T'Entregas é um Marketplace especializado em comercialização 
de comidas diversas, a empresa tem o conceito de entrega a onde você estive.\par

A empresa aqui descrita, tem o papel de encontrar empresas e pessoas que 
desejam come a comida de sua escolha e recebê-la em até 30 minutos, a fim de atender as expectativas de um consumidor ávido e disposto a pagar 
pelo pedido e a entrega, tendo a certeza de quem, como, e com que tipo de 
matéria-prima foi produzida. \par

A empresa está antenada com o que há de mais inovador nas três dimensões mais determinantes de uma organização: tecnológica, cultural e ambiental, por ser uma empresa totalmente virtual, o cliente tem acesso a ela em qualquer lugar que tenha internet, potencializando assim a presença da empresa na vida das pessoas, essas que não aceitam mais consumir produtos de empresas que não tem compromisso com os problemas sociais e ambientais, essa mudança cultural exige das empresas uma mudança de comportamento. \par

Nesse sentido, a T'Entregas sai na frente, pensando justamente em mitigar os impactos ambientais e combater qualquer exploração do trabalho, opta por produtores comprometidos com a sustentabilidade. \par

Atinente a esses valores que ganham cada vez mais espaço na sociedade contemporânea, estão os valores econômicos atrelados a eles, desta forma, investir em uma empresa busca-se atender uma demanda com valores pautados na sustentabilidade é se adiantar a um processo de mudanças de comportamento que não tem mais volta, no caso da T'Entregas que atua em espaço totalmente online, as vantagens são ainda mais acentuadas, uma empresa virtual tem seu custo de investimentos iniciais bem menor que uma empresa física, e por isso, quem investe nesse tipo de empresa tem seu retorno de investimento mais rápido, a T'Entregas tem sua taxa interna de retorno em 202\%, com o ponto de equilíbrio em apenas quatro meses. A empresa terá sua formação societária composta por dois sócios, sua sede física será na cidade de Fortaleza - CE na residência de um dos sócios, o que contribui para a diminuição dos custos de operação. \par

\subsection{Estrutura legal do empreendimento}

A T'Entregas será uma empresa enquadra na modalidade Empresa 
de Pequeno Porte (EPP) de responsabilidade limitada (LTDA) sendo um dos tipos 
de sociedade regida pelo Código Civil - Lei n° 10.406. De acordo com o Sebrae 
(2019), uma Sociedade Empresária Limitada é composta por dois sócios ou mais, 
que tem como objetivo explorar atividade econômica organizada para a produção 
ou circulação de bens e serviços. \par

Em uma LTDA todos os sócios têm direito a uma porcentagem nos lucros, sendo necessário estar em uma cláusula no contrato social, mas, para garantir estabilidade ao negócio, em caso de prejuízo os sócios não recebem lucro da empresa. \par

Quanto à fiscalização, é necessária uma cláusula específica, havendo essa cláusula, é permitida consulta à sociedade e seus lucros, em relação a responsabilidade dos sócios, se houver quebra de regras, o sócio pode ser excluído do negócio para não causar prejuízos maiores ao negócio. \par

Para registrar a empresa como Sociedade Limitada é necessário ir a Junta Comercial da cidade e solicitar inscrição aos órgãos regulatórios, sendo necessário inscrição na Receita Federal para obter o CNPJ, solicitar a Secretaria da Fazenda a inscrição estadual e o ICMS, e para o alvará de funcionamento, solicitar a autorização da prefeitura.\par

\subsection{Opção tributária}

A T'Entregas optou pelo Simples Nacional como regime tributário, sendo este previsto na Lei Complementar n° 123, de 14 de dezembro de 2006. Microempresas e Empresas de Pequeno Porte podem aderir ao Simples Nacional, sendo um regime compartilhado de arrecadação, cobrança e fiscalização de tributos. \par
Tributos que o Simples Nacional abrange: \par
Imposto de Renda da Pessoa Jurídica (IRPJ); \par
Contribuição Social sobre o Lucro Líquido (CSLL); \par
Programa de Integração Social (PIS/Pasep); \par
Contribuição para o Financiamento da Seguridade Social (Cofins); \par
Imposto sobre Produtos Industrializados; \par
Imposto sobre Circulação de Mercadorias e Serviços (ICMS); \par
Imposto Sobre Serviço (ISS); \par
Contribuição para a Seguridade Social destinada a Previdência Social a 
cargo da pessoa jurídica (CPP); \par

\subsection{Layout do site da Empresa}

\begin{commentB}
	FIGURA 25 - LAYOUT DA PRIMEIRA PÁGIANA DA PLATAFORM
	\par \end{commentB}

\subsection{Visão, Missão e Valores da empresa}

\noindent VISÃO \par
Se tornar referência na oferta de produtos de vestuário para consumo sustentável.\par
\noindent MISSÃO \par
Fortalecer e popularizar o consumo consciente, facilitando o acesso ao vestuário 
sustentável por intermédio de uma plataforma aglutinadora desses.\par
\noindent VALORES \par
Não compactuamos com qualquer forma de degradação;\par
Diálogo horizontal com quem faz e com quem consome;\par
Respeitamos a família independente de sua composição;\par
Trabalhamos com relação de ganha-ganha;\par
Não compactuamos com nenhuma exploração de mão-de-obra.\par
