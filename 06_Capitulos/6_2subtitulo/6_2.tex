\section{\textbf{O operacional}}
\label{sec: O operacional}

\begin{commentA}
	
	\par \end{commentA}

\begin{commentB}
	
	\par \end{commentB}

\subsection{Capacidade produtiva, comercial e de prestação de serviços.}
A capacidade produtiva aqui em questão, corresponde ao servidor em que a 
plataforma está hospedada e sua capacidade de suportar múltiplos acessos 
simultaneamente. O serviço de hospedagem será o Amazon Web Services (AWS). 
Esta capacidade será aumentada de acordo com que a plataforma for se expandindo.\par

\subsection{Processos operacionais}
Aqui se encontram as atividades realizadas pela empresa. A plataforma fará a 
intermediação de prestação de serviços onde, clientes finais e organizadores de 
eventos a utilizarão para procurar prestadores de serviços para sete categoriais, 
inicialmente: bolos e tortas, docinhos, salgadinhos, serviço de bebidas, serviço de buffet, serviço de decoração e serviço de garçom.\par

A grande vantagem da plataforma se dá na hora do pagamento, onde os 77
consumidores poderão escolher mais de um prestador de serviço e realizar apenas 
um pagamento. Este pagamento irá não diretamente para cada prestador de serviço 
contratado, mas sim para a empresa da plataforma. A empresa, então, repassará esse pagamento aos prestadores de serviço, porém, com o desconto da taxa de 
corretagem.\par

\subsection{Necessidade de pessoal}
Os colaboradores necessários para o bom funcionamento da plataforma são: 
• Desenvolvedor; 
• Analista de pós-vendas; 
• Analista de vendas; 
• Analista financeiro; 
• Auxiliar administrativo; e 
• Diretor. 
O desenvolvedor e analista de pós-vendas formarão a equipe de suporte em 
que o analista será o colaborador com o qual os usuários da plataforma terão contato, 
em caso de alguma dúvida ou problema, e o desenvolvedor será o funcionário que 
fará a manutenção da plataforma, fazendo os ajustes e testes necessários. 
O analista de vendas será responsável por atrair novos clientes à plataforma, 
com auxílio de uma agência de publicidade contratada. 
A grande função do analista financeiro é repassar o pagamento dos
consumidores aos prestadores de serviço, com o desconto da taxa de corretagem. 
O auxiliar administrativo será um estagiário, preferencialmente cursando 
Administração a partir da 4ª fase. Ele será responsável por auxiliar nas atividades de 
vendas e financeiras. 
O diretor tem a função de gerenciar a empresa como um todo e suas atividades 
administrativas próprias. 
Para o processo de recrutamento e seleção de pessoal, não será necessário 
contratar uma pessoa dedicada à função. Uma empresa terceirizada será contratada 
à medida que houver necessidade de novos funcionários.






Plano Operacional 

5.4.1. Capacidade produtiva, comercial e de prestação de serviços 
A capacidade produtiva aqui em questão, corresponde ao servidor em que a plataforma está hospedada e sua capacidade de suportar múltiplos acessos simultaneamente. O serviço de hospedagem será o Amazon Web Services (AWS). Esta capacidade será aumentada de acordo com que a plataforma for se expandindo.  

5.4.2. Processos operacionais 
Aqui se encontram as atividades realizadas pela empresa. A plataforma fará a intermediação de prestação de serviços onde, clientes finais e organizadores de eventos a utilizarão para procurar prestadores de serviços para sete categoriais, inicialmente: bolos e tortas, docinhos, salgadinhos, serviço de bebidas, serviço de buffet, serviço de decoração e serviço de garçom. 
A grande vantagem da plataforma se dá na hora do pagamento, onde os consumidores poderão escolher mais de um prestador de serviço e realizar apenas um pagamento. Este pagamento irá não diretamente para cada prestador de serviço contratado, mas sim para a empresa da plataforma. A empresa, então, repassará esse pagamento aos prestadores de serviço, porém, com o desconto da taxa de corretagem. 

5.4.3. Necessidade de pessoal 
Os colaboradores necessários para o bom funcionamento da plataforma são: 
•	Desenvolvedor; 
•	Analista de pós-vendas; 
•	Analista de vendas; 
•	Analista financeiro; • Auxiliar administrativo; e 
•	Diretor. 

O desenvolvedor e analista de pós-vendas formarão a equipe de suporte em que o analista será o colaborador com o qual os usuários da plataforma terão contato, em caso de alguma dúvida ou problema, e o desenvolvedor será o funcionário que fará a manutenção da plataforma, fazendo os ajustes e testes necessários. 
O analista de vendas será responsável por atrair novos clientes à plataforma, com auxílio de uma agência de publicidade contratada. 
A grande função do analista financeiro é repassar o pagamento dos consumidores aos prestadores de serviço, com o desconto da taxa de corretagem. 
O auxiliar administrativo será um estagiário, preferencialmente cursando Administração a partir da 4ª fase. Ele será responsável por auxiliar nas atividades de vendas e financeiras. 
O diretor tem a função de gerenciar a empresa como um todo e suas atividades administrativas próprias. 
Para o processo de recrutamento e seleção de pessoal, não será necessário contratar uma pessoa dedicada à função. Uma empresa terceirizada será contratada à medida que houver necessidade de novos funcionários. 
