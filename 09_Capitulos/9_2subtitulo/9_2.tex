\section{\textbf{9.2 Descrição Financeira}}
\label{sec: 9.2 Descrição Financeira}



\vspace*{0.3cm}
\begin{table}[!htbp]
	\raggedright.
	\caption{Faixa etária}
	\vspace*{-0.3cm}
	\pgfplotstableread[row sep=\\,col sep=&]{
		interval  & carR \\
		17        & 16.15  \\
		18--23    & 14.98  \\
		24--30    & 19.84 \\
		31--40    & 16.93 \\
		41--54    & 15.95 \\
		55+       & 16.15 \\
	}\mydata
	
	\begin{tikzpicture}
		\begin{axis}[
			ybar,
			bar width=.8cm,
			width=1\textwidth,
			height=0.5\textwidth,
			legend style={at={(0.5,1)},
				anchor=north,legend columns=-1},
			symbolic x coords={17,18--23,24--30,31--40,41--54,55+},
			xtick=data,
			nodes near coords,
			nodes near coords align={vertical},
			ymin=0,ymax=30,
			ylabel={Usuários\%},
			]
			%        \addplot table[x=interval,y=carT]{\mydata};
			%        \addplot table[x=interval,y=carD]{\mydata};
			\addplot table[x=interval,y=carR,fill=RYB4]{\mydata};
			\legend{Faixa de idade}
			
		\end{axis}
	\end{tikzpicture}
	\vspace*{-0.5cm}
	\par
	\bigskip
	\small  Compilado pelo SPSS da pesquisa de campo
\end{table}
\par





Packages may be found on 
\cooltooltip{CTAN}{A link to CTAN}{http://www.ctan.org/}{Visit CTAN on	the web}{CTAN}.

\cooltooltip
[0 0 1]
{Example}
{This is an example of a cool tooltip. Pretty cool, eh?}
{http://www.ctan.org/}{Visit CTAN on the Web}
{This text\strut}

\cooltooltip
[0 0 1]
{Example}
{This is an example of a cool tooltip. Pretty cool, eh?}
{http://www.ctan.org/}{Visit CTAN on the Web}
{This text\strut}

\cooltooltip [hpopup colori] [hlink colori]
{hsubjecti} {hmessagei} {hurli} {htooltipi} {htexti}


\cooltooltip [1 0 0]{Verificação de pacote}{~Este link abre meu GDrive~}{https://drive.google.com/drive/u/2/folders/1kXfYhZI4ohhAV-T8iRcpWZgwWniTf1Gb}{Google Drive David}{GDrive\strut}

\cooltooltiptoggle{\fcolorbox{black}{yellow}{Under no circumstances turn this knob.}}


\cooltooltiptoggle {\fcolorbox{black}{gray!30}{http://www.ctan.org/}}









		



\begin{filecontents}{conteudo.tex}
	File 1
\end{filecontents}

\begin{filecontents*}{conteudo.tex}
	File 2
\end{filecontents*}


