\section{\textbf{9.1 título}}
\label{sec: 9.1 Sub título}

Estimativa dos investimentos fixos\par
Na categoria de investimento fixo estão as compras dos notebooks utilizados 
pelos funcionários. Serão adquiridos 6 novos computadores, um para cada
colaborador. A especificação e investimento dos computadores se encontram abaixo:\par

\begin{commentB}
	Tabela 1 - Investimentos Fixos\\
	Investimento Quantidade Custo Unitário Custo Total\\
	Notebook Dell Inspiron 14 5000 6 2.459,00 R\$ 14.754,00
	\par \end{commentB}


Capital de giro 
Como capital de giro a empresa terá uma reserva inicial de segurança de R\$ 
10.000,00 (dez mil reais). \par

\begin{commentB}
	Tabela 2 - Capital de Giro
	Investimento Custo\\
	Reserva inicial 10.000,00 R\$ \\
	Capital de giro total 10.000,00
	\par \end{commentB}


Investimento pré-operacional \par
Para o investimento pré-operacional, ou seja, antes de a empresa entrar em 
atividade, o item primordial é o desenvolvimento da plataforma, que será realizado por 
uma empresa terceirizada. Esse tipo de empresa focada no desenvolvimento de 
software tem o nome de Software House. Em conversa com um desenvolvedor com 
experiência sênior na área, o valor do desenvolvimento da plataforma ficará em torno 
de R\$ 40.000,00. \par

\begin{commentB}
	Tabela 3 - Investimentos Pré-Operacionais\\
	Investimento Custo\\
	Desenvolvimento da plataforma R\$40.000,00
	\par \end{commentB}


Investimento total (resumo)\par
De acordo com os valores apresentados anteriormente, o investimento total é
de R\$ 64.754,00. \par

\begin{commentB}
	Tabela 4 - Investimento Total \\
	Investimento Custo \\
	Investimento Pré-Operacional 40.000,00 R\$ \\
	Capital de Giro 10.000,00 R\$ \\
	Investimento Fixo 14.754,00 R\$ \\
	Investimento Total 64.754,
	\par \end{commentB}


Estimativa do faturamento mensal \par
Através de pesquisas realizadas em sites de divulgação e avaliação de serviços 
(hagah, TripAdvisor e StarOfService) e consulta com o Sindicato de Hotéis, 
Restaurantes, Bares e Similares de Florianópolis (SHRBS), o autor deste trabalho 
formulou a seguinte tabela com a quantidade de prestadores de serviços para cada 
categoria. \par

\begin{commentB}
	Tabela 5 - Prestadores de Serviço por Categoria \\
	Categoria Prestadores de 
	Serviço Total \\
	Bolos e tortas 189 \\
	Docinhos 68 \\
	Salgadinhos 52 \\
	Serviço de bebidas 38 \\
	Serviço de buffet 77 \\
	Serviço de decoração 59 \\
	Serviço de garçom 46
	
	\par \end{commentB}



A partir desses dados, foi estimado que, no início das operações, 
aproximadamente 15\% dos prestadores de serviços das categorias “Bolos e tortas”, 
“Docinhos” e “Salgadinhos” estejam utilizando a plataforma. Já para as categorias de 
serviços, este número cai para 10\%, chegando a um total de 64 prestadores de serviços utilizando a plataforma no primeiro mês de atividade.\par

\begin{commentB}
	Tabela 6 - Prestadores de Serviço na Plataforma \\
	Categoria Prestadores de 
	Serviço Total 15\% 10\% \\
	Bolos e tortas 189 28 \\
	Docinhos 68 10 \\
	Salgadinhos 52 7 \\
	Serviço de bebidas 38 3  \\
	Serviço de buffet 77 7 \\
	Serviço de decoração 59 5 \\
	Serviço de garçom 46 4 \\
	Prestadores de Serviço Utilizando a Plataforma 64 \\
	\par \end{commentB}




A taxa de mudança de um mês para outro se dá por indicadores de crescimento 
empresariais, obtidos através de dados adquiridos com a Junta Comercial do Estado 
de Santa Catarina (JUCESC). Foram observados os últimos 5 anos (apêndice D) e 
feita uma média do saldo de abertura menos fechamento de empresas por mês, entre 
novembro de 2012 e outubro de 2017, conforme a tabela abaixo. Vale destacar a 
diferença apresentada em janeiro referente a dezembro de -66,18\%, apontando que 
no início de cada ano o número de empresas diminui. Porém, de janeiro para fevereiro, 
ocorre um aumento significativo de 124,12\%. \par

\begin{commentB}
	Tabela 7 - Média das Estatísticas Novembro de 2012 a Outubro de 2017
	\par \end{commentB}
