\chapter{Finanças}
\label{chapter: Finanças}


\begin{commentA}

\par \end{commentA}

Plano Financeiro 

5.5.1. Estimativa dos investimentos fixos 
Na categoria de investimento fixo estão as compras dos notebooks utilizados pelos funcionários. Serão adquiridos 6 novos computadores, um para cada colaborador. A especificação e investimento dos computadores se encontram abaixo: 

Tabela 1 - Investimentos Fixos 
Investimento	Quantidade	Custo Unitário	Custo Total
Notebook Dell Inspiron 14 5000	6	R$    2 .459,00	R$  14.754,00
Fonte: Elaborado pelo autor 

5.5.2. Capital de giro 
Como capital de giro a empresa terá uma reserva inicial de segurança de R\$ 
10.000,00 (dez mil reais).\par


Tabela 2 - Capital de Giro 
Investimento	Custo
Reserva inicial	R\$  10.000,00
Capital de giro total	R\$  10.000,00
Fonte: Elaborado pelo autor \par


5.5.3. Investimento pré-operacional 
Para o investimento pré-operacional, ou seja, antes de a empresa entrar em atividade, o item primordial é o desenvolvimento da plataforma, que será realizado por uma empresa terceirizada. Esse tipo de empresa focada no desenvolvimento de software tem o nome de Software House. Em conversa com um desenvolvedor com experiência sênior na área, o valor do desenvolvimento da plataforma ficará em torno de R\$ 40.000,00.\par




Tabela 3 - Investimentos Pré-Operacionais 
Investimento	Custo
Desenvolvimento da plataforma	R\$  40.000,00
Fonte: Elaborado pelo autor \par



5.5.4. Investimento total (resumo) 
De acordo com os valores apresentados anteriormente, o investimento total é de R\$ 64.754,00. \par


Tabela 4 - Investimento Total 
Investimento	Custo
Investimento Pré-Operacional	R\$  40.000,00
Capital de Giro	R\$  10.000,00
Investimento Fixo	R\$  14.754,00
Investimento Total	R\$  64.754,00
Fonte: Elaborado pelo autor \par


5.5.5. Estimativa do faturamento mensal 
Através de pesquisas realizadas em sites de divulgação e avaliação de serviços (hagah, TripAdvisor e StarOfService) e consulta com o Sindicato de Hotéis, Restaurantes, Bares e Similares de Florianópolis (SHRBS), o autor deste trabalho formulou a seguinte tabela com a quantidade de prestadores de serviços para cada categoria. 

Tabela 5 - Prestadores de Serviço por Categoria 
Categoria	Prestadores de Serviço Total
Bolos e tortas	189
Docinhos	68
Salgadinhos	52
Serviço de bebidas	38
Serviço de buffet	77
Serviço de decoração	59
Serviço de garçom	46
Fonte: Elaborado pelo autor 

A partir desses dados, foi estimado que, no início das operações, aproximadamente 15\% dos prestadores de serviços das categorias “Bolos e tortas”, “Docinhos” e “Salgadinhos” estejam utilizando a plataforma. Já para as categorias de serviços, este número cai para 10\%, chegando a um total de 64 prestadores de serviços utilizando a plataforma no primeiro mês de atividade. 

Tabela 6 - Prestadores de Serviço na Plataforma 
Categoria	Prestadores de Serviço Total	15\%	10\%
Bolos e tortas	189	28	
Docinhos	68	10	
Salgadinhos	52	7	
Serviço de bebidas	38		3
Serviço de buffet	77		7
Serviço de decoração	59		5
Serviço de garçom	46		4
Prestadores de Serviço Utilizando a Plataforma	64
Fonte: Elaborado pelo autor 

A taxa de mudança de um mês para outro se dá por indicadores de crescimento empresariais, obtidos através de dados adquiridos com a Junta Comercial do Estado de Santa Catarina (JUCESC). Foram observados os últimos 5 anos (apêndice D) e feita uma média do saldo de abertura menos fechamento de empresas por mês, entre novembro de 2012 e outubro de 2017, conforme a tabela abaixo. Vale destacar a diferença apresentada em janeiro referente a dezembro de -66,18\%, apontando que no início de cada ano o número de empresas diminui. Porém, de janeiro para fevereiro, ocorre um aumento significativo de 124,12\%. 

Tabela 7 - Média das Estatísticas Novembro de 2012 a Outubro de 2017 
Constituições	Extinções	Falência	Total	Diferença \%
Jan	1282	1127	3	152	-66,18\%
Fev	1674	1331	2	341	124,12\%
Mar	1997	1510	2	486	42,56\%
Abr	1854	1272	1	581	19,56\%
Mai	1907	1396	2	510	-12,26\%
Jun	2081	1228	1	853	67,39\%
Jul	2241	1360	1	880	3,13\%
Ago	2069	1268	7	794	-9,70\%
Set	2069	1256	2	811	2,16\%
Out	2052	1230	3	819	0,94\%
Nov	1704	1172	1	531	-35,19\%
Dez	1337	885	3	449	-15,34\%
Total	22266	15034	26	7206	
Fonte: Elaborado pelo autor 

Com base nesses valores de mudança de um mês para outro, a seguinte tabela foi formada, apresentando a quantidade de prestadores de serviços cadastrados na plataforma, sendo o primeiro mês de atividade equivalente ao mês de janeiro. 

Tabela 8 - Acumulado de Prestadores de Serviço Cadastrados Ano 1 
Mês	Acumulado de Prestadores de Serviço Cadastrados
1	64
2	143
3	205
4	246
5	216
6	362
7	372
8	335
9	342
10	345
11	224
12	191
Fonte: Elaborado pelo autor 

Tabela 9 - Acumulado de Prestadores de Serviço Cadastrados Ano 2 
Mês	Acumulado de Prestadores de Serviço Cadastrados
13	65
14	145
15	208
16	249
17	219
18	366
19	377
20	340
21	346
22	350
23	227
24	193
Fonte: Elaborado pelo autor 

Tabela 10 - Acumulado de Prestadores de Serviço Cadastrados Ano 3 
Mês	Acumulado de Prestadores de Serviço Cadastrados
25	66
26	147
27	211
28	253
29	222
30	371
31	382
32	344
33	351
34	355
35	230
36	196
Fonte: Elaborado pelo autor 

O número de atendimentos mensais e o preço médio das vendas, foram passados pelos empresários nas entrevistas realizadas. Estimou-se que, desses atendimentos, 1/3 sejam realizados através da plataforma. 

Tabela 11 - Atendimento Mensais 
Categoria	Atendimentos Mensais	 1/3
Bolos e tortas	28	9
Docinhos	40	13
Salgadinhos	40	13
Serviço de bebidas	30	10
Serviço de buffet	21	7
Serviço de decoração	21	7
Serviço de garçom	25	8
Fonte: Elaborado pelo autor 

Após tomar conhecimento das informações passadas anteriormente, como os três primeiros meses serão gratuitos e a forma de cobrança apontada no item 5.3.2 de 10% sobre a venda, pôde-se elaborar uma previsão de faturamento para 36 meses, executando a seguinte fórmula elaborada pelo autor: 


FM = (nPS x $ x nAM) x TX, onde: 
•	FM: faturamento mensal; 
•	nPS: número de prestadores de serviço; 
•	$: preço médio da venda; 
•	nAM: número de atendimentos mensais; e 
•	TX: taxa de corretagem (10%) 

Foi feita a previsão para 36 meses de cada categoria independente e depois somado para obter o faturamento mensal total. As estimativas de faturamento de cada categoria se encontram no apêndice E. Já o faturamento total mensal, para os primeiros 36 meses, está abaixo: 

Tabela 12 - Faturamento Mensal Ano 1 
Mês	Acumulado de Prestadores de Serviços Pagantes	Faturamento Mensal
1	0	R\$                 -
2	0	R\$                 -
3	0	R\$                 -
4	64	R\$     10.729,33
5	143	R\$     24.033,71
6	205	R\$     34.368,20
7	246	R\$     41.241,84
8	216	R\$     36.292,82
9	362	R\$     60.609,01
10	372	R\$     62.427,28
11	335	R\$     56.184,55
12	342	R\$     57.308,24
Faturamento Total Anual	R\$  383.194,98
Fonte: Elaborado pelo autor 

Tabela 13 - Faturamento Mensal Ano 2 
Mês	Acumulado de Prestadores de Serviços Pagantes	Faturamento Mensal
13	345	R\$     57.881,32
14	224	R\$     37.622,86
15	191	R\$     31.979,43
16	65	R\$     10.873,01
17	145	R\$     24.355,54
18	208	R\$     34.828,42
19	249	R\$     41.794,10
20	219	R\$     36.778,81
21	366	R\$     61.420,61
22	377	R\$     63.263,23
23	340	R\$     56.936,90
24	346	R\$     58.075,64
Faturamento Total Anual	R\$  515.809,86
Fonte: Elaborado pelo autor 

Tabela 14 - Faturamento Mensal Ano 3 
Mês	Acumulado de Prestadores de Serviços Pagantes	Faturamento Mensal
25	350	R\$     58.656,40
26	227	R\$     38.126,66
27	193	R\$     32.407,66
28	66	R\$     11.018,60
29	147	R\$     24.681,67
30	211	R\$     35.294,79
31	253	R\$     42.353,75
32	222	R\$     37.271,30
33	371	R\$     62.243,07
34	382	R\$     64.110,37
35	344	R\$     57.699,33
36	351	R\$     58.853,32
Faturamento Total Anual	R\$  522.716,93
Fonte: Elaborado pelo autor 

5.5.6. Estimativa dos custos de comercialização 
Os custos de comercialização presentes neste tipo de negócio são os impostos aplicados sobre as vendas. Como o faturamento anual dos 3 anos de operação fica entre R\$ 360.000,00 e R\$ 540.000,00, a alíquota total é de 10,26\%. 

5.5.7. Estimativa dos custos com mão-de-obra 
O custo mensal para cada cargo está apresentado na tabela abaixo, conforme vigência de empresas optantes pelo Simples Nacional. Foi atribuída uma taxa de margem de segurança, com valor estipulado em 25\% dos custos adicionais, para eventuais acontecimentos, como em caso de hora extra e substituição de funcionário. 
O custo total mensal com funcionários é de R\$ 19.712,20. 

Tabela 15 - Custo de Mão-de-Obra 
Desenvolvedor	Analista de Pós-Vendas	Analista de Vendas	Analista Financeiro	Auxiliar Administrativo
Salário Base	R\$     2.800,00	R\$  2.500,00	R\$  2.300,00	R\$  2.300,00	R\$   1.200,00
Encargos Trabalhistas		
FGTS Salário	R\$        224,00	R\$     200,00	R\$     184,00	R\$     184,00	R\$       96,00
Férias (1/12)	R\$        233,33	R\$     208,33	R\$     191,67	R\$     191,67	R\$     1 00,00
1/3 Férias (1/12)	R\$         7 7,78	R\$       69,44	R\$       63,89	R\$       63,89	R\$       33,33
13º Salário (1/12)	R\$        233,33	R\$     208,33	R\$     191,67	R\$     191,67	R\$     1 00,00
FGTS Férias (1/12)	R\$         1 8,67	R\$       16,67	R\$       15,33	R\$       15,33	R\$         8,00
FGTS 1/3 Férias (1/12)	R\$           6,16	R\$        5,50	R\$        5,06	R\$        5,06	R\$         2,64
FGTS 13º Salário (1/12)	R\$         1 8,67	R\$       16,67	R\$       15,33	R\$       15,33	R\$         8,00
Total Encargos	R\$        811,94	R\$     724,94	R\$     666,95	R\$     666,95	R\$     3 47,97
Provisões		
Aviso Prévio (1/12)	R\$        233,33	R\$     208,33	R\$     191,67	R\$     191,67	R\$     1 00,00
FGTS Aviso Prévio (1/12)	R\$         1 8,67	R\$       16,67	R\$       15,33	R\$       15,33	R\$         8,00
Multa FGTS (1/12)	R\$         5 6,00	R\$       50,00	R\$       46,00	R\$       46,00	R\$       24,00
Total Provisões	R\$        308,00	R\$     275,00	R\$     253,00	R\$     253,00	R\$     1 32,00
Benefícios		
Auxílio Refeição	R\$        320,00	R\$     320,00	R\$     320,00	R\$     320,00	R\$     3 20,00
Auxílio Transporte	R\$        170,00	R\$     170,00	R\$     170,00	R\$     170,00	R\$     1 70,00
Total Benefícios	R\$        490,00	R\$     490,00	R\$     490,00	R\$     490,00	R\$     4 90,00
Total Adicional	R\$     1.609,94	R\$  1.489,94	R\$  1.409,95	R\$  1.409,95	R\$      969,97
Margem de Segurança	R\$        402,48	R\$     372,49	R\$     352,49	R\$     352,49	R\$     2 42,49
Custo Total	R\$     4.812,42	R\$  4.362,43	R\$  4.062,44	R\$  4.062,44	R\$   2.412,47
Fonte: Elaborado pelo autor 

5.5.8. Estimativa dos custos com depreciação 
Enquanto estiver no escritório compartilhado, apenas haverá depreciação com os computadores da empresa. Após começar a utilizar um escritório particular, haverá depreciação com computadores, mesas e cadeiras. A plataforma desenvolvida antes do início das operações da empresa, no valor de R\$ 40.000,00, é um ativo intangível com vida útil indefinida e, de acordo com o Comitê de Pronunciamentos Contábeis (CPC), em seu pronunciamento CPC 04, “um ativo intangível com vida útil indefinida não deve ser amortizado” (CPC 04, 2010). 
Os notebooks possuem vida útil de 5 anos, com isso, apresentam uma taxa de depreciação de 20\% ao ano. Portanto, de acordo com os valores dos notebooks apresentados anteriormente, segue abaixo o valor com a depreciação anual. 

Tabela 16 - Custos Depreciação 
Produto	Quantidade	Depreciação Anual
Unitária	Total
Notebook Dell Inspiron 14 5000	6	R\$  491,80	R\$  2.950,80
Fonte: Elaborado pelo autor 

5.5.9. Estimativa dos custos fixos operacionais mensais 
Nesta categoria entram os custos necessários para o funcionamento operacional da empresa: aluguel do escritório, servidor, campanhas de marketing, salários e pró-labore. 

Tabela 17 - Custos Fixos Operacionais Mensais 
Descrição	Custo
Aluguel	R\$    1 .500,00
Servidor	R\$        300,00
Marketing	R\$        500,00
Salários	R\$  19.712,19
Pró-labore	R\$    3 .000,00
Custo Total	R\$  25.012,19
Fonte: Elaborado pelo autor 

5.5.10. 	Demonstração de Resultado do Exercício (DRE) 
O demonstrativo de resultados é um relatório de caráter financeiro. No apêndice F, encontram-se os demonstrativos detalhados dos 36 primeiros meses de operação. Abaixo, segue a Demonstração de Resultados do Exercício (DRE) dos 3 anos iniciais. 

Tabela 18 - Demonstração de Resultados do Exercício 
DRE	Ano 1	Ano 2	Ano 3
Receita bruta	R\$  3 83.194,98	R\$   515.809,86	R\$   522.716,93
(-) Deduções e abatimentos	R\$     39.315,81	R\$     52.922,09	R\$     53.630,76
(=) Receita Líquida	R\$  3 43.879,18	R\$   462.887,77	R\$   469.086,17
(-) Custos Fixos	R\$     27.600,00	R\$     27.600,00	R\$     27.600,00
(-) Custos de Mão-de-Obra	R\$  2 36.546,34	R\$   236.546,34	R\$   236.546,34
(-) Depreciação Acumulada	R\$       2.950,80	R\$       2.950,80	R\$       2.950,80
(=) Lucro Líquido Antes das 
Participações	R\$     76.782,04	R\$   195.790,63	R\$   201.989,03
(-) Pró Labore	R\$     36.000,00	R\$     36.000,00	R\$     36.000,00
(=) Resultado Líquido do 
Exercício	R\$     40.782,04	R\$   159.790,63	R\$   165.989,03
Fonte: Elaborado pelo autor 

5.5.11. 	Indicadores de viabilidade 
Para os indicadores de viabilidade do negócio, temos: ponto de equilíbrio, lucratividade, rentabilidade e o prazo de retorno do investimento. 

5.5.11.1. 	Ponto de equilíbrio 
O ponto de equilíbrio ocorre no momento em que as receitas e as despesas são iguais. Para chegar neste ponto, devemos subtrair as despesas das receitas e encontrar os lucros e prejuízos acumulados até então. Como mencionado anteriormente, a empresa investiu R\$ 40.000,00 no desenvolvimento da plataforma. Esse investimento aparece no mês 0. O ponto de equilíbrio ocorre no mês 12, como podemos observar abaixo: 




10%, visto que existe um grande investimento com a plataforma. Já no segundo e terceiro anos, a empresa opera com lucratividade acima de 30%. 

Tabela 20 - Índice de Lucratividade 
Ano 1	Ano 2	Ano 3
Lucro Líquido	R\$     40.782,04	R\$  1 59.790,63	R\$   165.989,03
Receita	R\$   383.194,98	R\$  5 15.809,86	R\$   522.716,93
Lucratividade	10,64\%	30,98\%	31,76\%
Fonte: Elaborado pelo autor 

5.5.11.3. 	Rentabilidade 
Este índice representa o retorno de capital investido ao empreendedor. Ele é calculado dividindo o lucro líquido anual pelo investimento total. A seguir temos o índice de rentabilidade para os 3 primeiros anos de operação.  

Tabela 21 - Índice de Rentabilidade 
Ano 1	Ano 2	Ano 3
Lucro Líquido	R\$     40.782,04	R\$   159.790,63	R\$   165.989,03
Investimento	R\$   145.000,00	R\$   145.000,00	R\$   145.000,00
Rentabilidade	28,13\%	110,20\%	114,48\%
Fonte: Elaborado pelo autor 

5.5.11.4. 	Prazo de retorno do investimento (PRI) 
Este prazo informa o tempo necessário para que se recupere tudo que foi investido no negócio. O retorno do investimento total de R\$ 145.000,00 ocorrerá ao final do mês 24, ou seja, o prazo de retorno do investimento é de 2 anos. 

90 

Tabela 22 - Prazo de Retorno do Investimento 
Mês	Lucro e Prejuízo Acumulado
21	R\$        76.362,28
22	R\$     107.876,60
23	R\$     133.713,69
24	R\$     160.572,67
25	R\$     187.952,83
26	R\$     196.909,60
27	R\$     200.734,14
Fonte: Elaborado pelo autor 

5.6. 	Construção de Cenários 
O cenário tratado até então é considerado o cenário realista, visto que se baseou em dados reais obtidos pelo autor. É necessário, porém, criar outros dois cenários, um pessimista e ou otimista, para se ter uma melhor previsão para o negócio.  
Para a construção dos cenários, levou-se em conta uma variação de 20\% sobre as receitas, mantendo as despesas iguais ao cenário realista. 

5.6.1. Cenário pessimista 
Neste cenário, estipulou-se uma redução de 20\% sobre as receitas. Com isso, a DRE para os três anos é a seguinte: 

Tabela 23 - DRE Simulada no Cenário Pessimista 
DRE	Ano 1	Ano 2	Ano 3
Receita bruta	R\$   306.555,99	R\$   412.647,89	R\$   418.173,54
(-) Deduções e abatimentos	R\$     31.460,31	R\$     52.922,09	R\$     53.630,76
(=) Receita Líquida	R\$   275.095,68	R\$   359.725,80	R\$   364.542,78
(-) Custos Fixos	R\$     27.600,00	R\$     27.600,00	R\$     27.600,00
(-) Custos de Mão-de-Obra	R\$   236.546,34	R\$   236.546,34	R\$   236.546,34
(-) Depreciação Acumulada	R\$       2.950,80	R\$       2.950,80	R\$       2.950,80
(=) Lucro Líquido Antes das 
Participações	R\$       7.998,54	R\$     92.628,66	R\$     97.445,65
(-) Pró Labore	R\$     36.000,00	R\$     36.000,00	R\$     36.000,00
(=) Resultado Líquido do 
Exercício	-R\$     28.001,46	R\$     56.628,66	R\$     61.445,65
Fonte: Elaborado pelo autor 

Enquanto no cenário realista o lucro líquido do primeiro ano é de R\$ 40.782,04, no cenário pessimista este valor é muito inferior, passando de R\$ 28.000,00 negativos. 
O índice de lucratividade do cenário pessimista começa negativo, porém se recupera no ano 2. Segue abaixo para os três primeiros anos:  

Tabela 24 - Índice de Lucratividade Pessimista 
Ano 1	Ano 2	Ano 3
Lucro Líquido	-R\$     28.001,46	R\$     56.628,66	R\$     61.445,65
Receita	R\$   306.555,99	R\$   412.647,89	R\$   418.173,54
Lucratividade	-9,13\%	13,72\%	14,69\%
Fonte: Elaborado pelo autor 

Com um lucro líquido inferior ao cenário realista, temos o seguinte índice de rentabilidade: 

Tabela 25 - Índice de Rentabilidade Pessimista 
Ano 1	Ano 2	Ano 3
Lucro Líquido	-R\$     28.001,46	R\$     56.628,66	R\$     61.445,65
Investimento	R\$   145.000,00	R\$   145.000,00	R\$   145.000,00
Rentabilidade	-19,31\%	39,05\%	42,38\%
Fonte: Elaborado pelo autor 

Utilizando a média do resultado líquido dos três anos (R\$ 30.024,28) é obtido o prazo do retorno do investimento: 4 anos e 10 meses. 

5.6.2. Cenário otimista 
Já no cenário otimista, estipulou-se um aumento de 20\% sobre as receitas. O negócio se torna lucrativo a partir do mês 10, do primeiro ano. 
A DRE simulada neste cenário para os três primeiros anos é a seguinte: 

92 

Tabela 26 - DRE Simulada no Cenário Otimista 
DRE	Ano 1	Ano 2	Ano 3
Receita bruta	R\$   459.833,98	R\$   618.971,83	R\$   627.260,31
(-) Deduções e abatimentos	R\$     39.315,81	R\$     58.338,10	R\$     59.119,28
(=) Receita Líquida	R\$   420.518,17	R\$   560.633,74	R\$   568.141,03
(-) Custos Fixos	R\$     27.600,00	R\$     27.600,00	R\$     27.600,00
(-) Custos de Mão-de-Obra	R\$   236.546,34	R\$   236.546,34	R\$   236.546,34
(-) Depreciação Acumulada	R\$       2.950,80	R\$       2.950,80	R\$       2.950,80
(=) Lucro Líquido Antes das 
Participações	R\$   153.421,04	R\$   293.536,60	R\$   301.043,89
(-) Pró Labore	R\$     36.000,00	R\$     36.000,00	R\$     36.000,00
(=) Resultado Líquido do 
Exercício	R\$   117.421,04	R\$  2 57.536,60	R\$   265.043,89
Fonte: Elaborado pelo autor 

Com ótimos resultados, o negócio nesta condição apresenta um bom índice de lucratividade logo no início da operação. 

Tabela 27 - Índice de Lucratividade Otimista 
Ano 1	Ano 2	Ano 3
Lucro Líquido	R\$   117.421,04	R\$   257.536,60	R\$   265.043,89
Receita	R\$   459.833,98	R\$   618.971,83	R\$   627.260,31
Lucratividade	25,54\%	41,61\%	42,25\%
Fonte: Elaborado pelo autor 

Apresentando um lucro líquido muito superior aos outros dois cenários, o índice de rentabilidade no ano 1 é acima de 80\%, como podemos ver a seguir: 

Tabela 28 - Índice de Rentabilidade Otimista 
Ano 1	Ano 2	Ano 3
Lucro Líquido	R\$   117.421,04	R\$   257.536,60	R\$   265.043,89
Investimento	R\$   145.000,00	R\$   145.000,00	R\$   145.000,00
Rentabilidade	80,98\%	177,61\%	182,79\%
Fonte: Elaborado pelo autor 

Utilizando a média do resultado líquido dos três anos (R\$ 30.024,28) é obtido o prazo do retorno do investimento: 9 meses

