\section{\textbf{Análise do setor}}
\label{sec: Análise do setor}

\begin{commentA} \vspace{0.3cm} \noindent Análise do ambiente de modelo de negócios (contexto, direcionadores e restrições): variáveis ambientais e do setor, tendências, organização etc. que impactam na geração de receitas e/ou na estrutura de custos. \par \vspace{0.1cm} \end{commentA}

\begin{commentA} \vspace{0.3cm} \noindent Quais são as tendências no setor de ......? \par \vspace{0.1cm} \end{commentA}

Os pedidos para entrega de comida nas duas primeiras semanas de março aumentaram 77\%; ao mesmo tempo, o número de parceiros de entregas e restaurantes cadastrados no aplicativo também aumentou. Desde que as ordens de quarentena foram emitidas em várias cidades e estados para conter a disseminação do coronavírus no Brasil, lojas, restaurantes, lojas e até mesmo prestadores de serviços viram seus clientes desaparecerem. As empresas devem reduzir os controles de acesso ou suspender os serviços, mesmo que temporariamente, porque uma das principais diretrizes das autoridades é que as pessoas devem ficar em casa. Por outro lado, é por isso que algumas pessoas têm visto sua demanda crescer \(e muito\)\: quem trabalha na entrega. (LARGHI, 2020)\par

\begin{commentA} \vspace{0.3cm} \noindent Qual o tamanho do mercado do setor de ....? \par \vspace{0.1cm} \end{commentA}

O setor de food service, que engloba todo alimento consumido fora de casa, vem se tornando peça fundamental para a economia brasileira – só em 2018, foram movimentados R\$ 205 bilhões no país, segundo o Instituto Foodservice Brasil. Poucas décadas atrás, por volta dos anos 1980, o delivery no Brasil era praticamente dominado por pizzarias. O primeiro a olhar para essa oportunidade foi o iFood, que permaneceu sem competidores por alguns anos – até 2016, quando a americana Uber Eats, que já tinha fincado raízes sólidas no Brasil com o serviço de transporte de passageiros, importou dos EUA seu serviço de entrega de comida. Talvez tenha faltado a eles jogo de cintura para pivotar e criar um serviço que não existia antes. (IODICE, 2019)\par

\begin{commentA} \vspace{0.3cm} \noindent Qual o número de consumidores do setor de ....? \par \vspace{0.1cm} \end{commentA}

Segundo dados da Associação Brasileira de Comércio Eletrônico (AbComm), as compras em supermercados online aumentaram 180\% desde que a Organização Mundial da Saúde (OMS) listou o Brasil como um dos países afetados pela Covid-19 em março. A Ebit | Nielsen revelou que, embora os consumidores de e-mercearia sejam considerados um setor essencial, seu número dobrou, portanto, não há necessidade de fechar a porta durante a pandemia. (FREITAS, 2020)\par

\begin{commentA} \vspace{0.3cm} \noindent Qual os números de estabelecimentos do setor de ....? \par \vspace{0.1cm} \end{commentA}

O número de estabelecimentos nessas outras verticais triplicou no mesmo período de comparação. A James investiu em infraestrutura para velocidade de processamento e em contratação de mais separadores de pedidos. Mas a James prefere não falar em porcentagens. (FONSECA, 2020)\par

\begin{commentA} \vspace{0.3cm} \noindent Qual o tamanho das vendas do setor de ....? \par \vspace{0.1cm} \end{commentA}

Pesquisa nacional realizada pelo Instituto de Locomotivas comissionada pela VR Benefícios mostra que 81\% dos estabelecimentos comerciais brasileiros começaram a prestar serviços durante a pandemia e manterão esse modelo. Anteriormente, apenas 49\% dos restaurantes, lanchonetes, padarias e mercados realizavam entrega em domicílio. (E-COMMERCE BRASIL, 2021)

\begin{commentA} \vspace{0.3cm} \noindent Quais são as projeções desse setor? \par \vspace{0.1cm} \end{commentA}

A pesquisa realizada pelo Food Consulting, parceira do Sebrae, em abril de 2020, com base no entendimento dos hábitos "alimentares", atitudes e personalidade dos consumidores, fornece elementos para aprimorar a tomada de decisões e ações de empresários e gestores que atuam em empresas de food service. No contexto da Covid-19, "Comer Fora de casa" e "Pedir por Delivery". (R. DOS SANTOS, 2020)

\begin{commentA} \vspace{0.3cm} \noindent Quais são o tamanho do setor? \par \vspace{0.1cm} \end{commentA}

No caso da quarentena forçada devido ao novo coronavírus, as empresas de entrega expressa estão entre as estrelas. De refeições leves à compra de material de limpeza ou alimentos em supermercados, a demanda aumentou. De acordo com dados da empresa de inteligência Compre \& Trust, as compras online de alimentos e bebidas aumentaram 339\% só em maio, três vezes mais que 133\% do comércio eletrônico total. (RIVEIRA, 2020)

\begin{commentA} \vspace{0.3cm} \noindent Quais a taxas de crescimento do setor? \par \vspace{0.1cm} \end{commentA}


\begin{commentA} \vspace{0.3cm} \noindent Como será o mercado de ......, nos próximos anos? \par \vspace{0.1cm} \end{commentA}

Embora a maioria das empresas de serviços de alimentação tenha investido em colaboração com grandes startups de entrega para aproveitar as vendas durante a nova pandemia de coronavírus, ainda existem alguns que escolheram outra rota de entrega do próprio aplicativo. Estreitar o relacionamento com os clientes, garantir que os pedidos fiquem íntegros e reduzir custos, pois os encargos de mercado podem representar até 30\% do valor das vendas, que é a principal motivação de quem aposta na tecnologia. (ZANATTA, 2021)

\begin{commentA} \vspace{0.3cm} \noindent Quais os fatores (comportamentais, demográficos, culturais, econômicos, etc.) que influenciam as projeções? \par \vspace{0.1cm} \end{commentA}


\begin{commentA} \vspace{0.3cm} \noindent Por que esse mercado é promissor? Rentabilidade/lucratividade. \par \vspace{0.1cm} \end{commentA}


\begin{commentA} \vspace{0.3cm} \noindent Por que esse mercado é promissor? Tamanho do mercado. \par \vspace{0.1cm} \end{commentA}


\begin{commentA} \vspace{0.3cm} \noindent Por que esse mercado é promissor? Taxa de crescimento. \par \vspace{0.1cm} \end{commentA}


\begin{commentA} \vspace{0.3cm} \noindent Por que esse mercado é promissor? Número de concorrentes. \par \vspace{0.1cm} \end{commentA}


\begin{commentA} \vspace{0.3cm} \noindent Por que esse mercado é promissor? Volume de clientes, etc... \par \vspace{0.1cm} \end{commentA}


\begin{commentA} \vspace{0.3cm} \noindent Como o mercado de .... ? está estruturado e segmentado? \par \vspace{0.1cm} \end{commentA}


\begin{commentA} \vspace{0.3cm} \noindent Quais as oportunidades e os riscos do mercado? \par \vspace{0.1cm} \end{commentA}


\begin{commentA} \vspace{0.3cm} \noindent * \par \vspace{0.1cm} \end{commentA}
