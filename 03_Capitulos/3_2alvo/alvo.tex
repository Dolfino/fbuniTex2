\section{\textbf{Mercado alvo}}
\label{sec: Mercado}

\begin{commentA} \vspace{0.3cm} \noindent Análise do público-alvo (padrões de comportamento): pesquisa com dados secundários do setor e fora dele, entrevistas com clientes potenciais, etnografia do cliente, uso do produto/serviços, grupo focal, degustação pelo cliente. \par \vspace{0.1cm} \end{commentA}

\begin{commentA} \vspace{0.3cm} \noindent Construção de protótipos da ideia (produto viável mínimo) e testagem: esboços de guardanapos, ad-libs, Canvas de Proposta de valor. \par \vspace{0.1cm} \end{commentA}

\begin{commentA} \vspace{0.3cm} \noindent Qual o perfil do cliente? \par \vspace{0.1cm} \end{commentA}

A análise de mercado foi feita através de questionários com o intuito de conhecer as preferências do público-alvo, apresentar as propostas, saber se o negócio tem viabilidade na região metropolitana de Fortaleza, e também conhecer as falhas dos nossos concorrentes diretos, como Ifood, Uber eats e Rappi para usarmos isso ao nosso favor.\par

A seguir estão os resultados de uma pesquisa de mercado realizada por meio do formulário de pesquisa on-line Google Forms, que tem como objeto 514 clientes potenciais, cobrindo a região metropolitana de Fortaleza. A pesquisa foi realizada em 3, 4 e 5 de dezembro de 2020. Esses gráficos mostram informações sobre clientes em potencial sob investigação. Esses estudos são muito importantes para estabelecer as estratégias de projeto, implantação e manutenção da empresa.\par

\begin{commentB}
Quadro: Faixa etária dos entrevistados.
\par \end{commentB}

\vspace*{0.3cm}
\begin{figure}[!htbp]
\begin{footnotesize}
\raggedright
\captionsetup{textfont=bf, labelfont=bf, font=footnotesize, justification=centering}
\caption{Faixa etária.} \label{Faixa etária}
	\vspace*{-0.3cm}    
			\begin{tikzpicture}
				\begin{axis}[
					symbolic x coords={-17,,18-23,,24-30,,31-40,,41-54,,55+,},
					xticklabel style={rotate=45,anchor=north east},
					xtick={-17,18-23,24-30,31-40,41-54,55+},
					ylabel=Usuários \%,
					ymajorgrids,
					bar width=0.8cm,
					width=1\textwidth,
					height=0.4\textwidth,
					legend style={at={(0.5,1)},
					anchor=north,legend columns=-1},
					nodes near coords,
					nodes near coords align={vertical},
					ymin=0,ymax=25,
					]
					\addplot[ybar,fill=RYB4] 
					coordinates {(-17,16.15)};
					
					\addplot[ybar,fill=RYB4] 
					coordinates {(18-23,14.98)};
					
					\addplot[ybar,fill=RYB4] 
					coordinates {(24-30,19.84)};
					
					\addplot[ybar,fill=RYB4] 
					coordinates {(31-40,16.93)};
					
					\addplot[ybar,fill=RYB4] 
					coordinates {(41-54,15.95)};
					
					\addplot[ybar,fill=RYB4] 
					coordinates {(55+,16.15)};					
				\end{axis}
	\end{tikzpicture}%}}
	\vspace*{-0.5cm}
	\par
	\bigskip
Fonte: Compilado pelo SPSS da pesquisa de campo
\end{footnotesize}
\end{figure} 
\par

Percebe-se que no gráfico apresentado existe um equilíbrio entre os públicos-alvo da empresa nas duas faixas etárias. Os dados recolhidos a partir desta questão destacam que o maior público tem entre 36 e 45 anos, respondendo por 27,50\% do quadro, respondendo por 27\%, seguido do público entre 14 e 25 anos, respondendo por 27\%. Concluímos que o público-alvo mais interessado em assinar os serviços da T'Entregas são jovens e adultos, talvez porque são melhores no parto e querem mais tempo.\par

\begin{commentB}
Quadro: Renda familiar.
\par \end{commentB}

\vspace*{0.3cm}
\begin{figure}[!htbp]
\begin{footnotesize}
	\raggedright
\captionsetup{textfont=bf, labelfont=bf, font=footnotesize, justification=centering}
	\caption{Renda familiar.} \label{Renda}
	\vspace*{-0.3cm}    
	\begin{tikzpicture}
		\begin{axis}[
			ybar,
			enlargelimits=0.15,
			legend style={at={(0.5,-0.2)},
				anchor=north,legend columns=-1},
			ylabel={Usuários \%},
			symbolic x coords={Até R\$ 2.090,,De R\$ 2.090 a R\$ 4.180,,De R\$ 4.180 a R\$ 6.270,,De R\$ 6.270 a R\$ 8.360,,De R\$ 8.360 a R\$ 10.450,,Acima de R\$ 10.450,},
			xtick={Até R\$ 2.090,De R\$ 2.090 a R\$ 4.180,De R\$ 4.180 a R\$ 6.270,De R\$ 6.270 a R\$ 8.360,De R\$ 8.360 a R\$ 10.450,Acima de R\$ 10.450,},
			nodes near coords,
			nodes near coords align={vertical},
			x tick label style={rotate=25,anchor=east},
			ymajorgrids,
			bar width=0.8cm,
			width=1\textwidth,
			height=0.35\textwidth,
			ymin=0,ymax=35,
			]
			\addplot [ybar,fill=RYB4] 
			coordinates {(Até R\$ 2.090,32.10) (De R\$ 2.090 a R\$ 4.180,33.66) (De R\$ 4.180 a R\$ 6.270,27.24) (De R\$ 6.270 a R\$ 8.360,2.53) (De R\$ 8.360 a R\$ 10.450,1.17) (Acima de R\$ 10.450,3.31)};
		\end{axis}
	\end{tikzpicture}%}}
	\vspace*{-0.5cm}
	\par
	\bigskip
Fonte: Compilado pelo SPSS da pesquisa de campo
\end{footnotesize}
\end{figure} 
\par
\FloatBarrier

A Figura 2 representa a renda familiar, 40\% das pessoas recebem de R \$ 1.672,00 a R \$ 1.628,00 a R \$ 2.628,00, 28,50\% de R \$ 2.629,00 a R \$ 3.628,00, 5,0\% e 4,50\% recebem de R \$ 4.629,00 a R \$ 6.628,00 ou mais, Contém 2\%. Percebe-se pela figura que a renda familiar do público mais interessado em uma alimentação saudável é diferente. Isso se reflete nas seguintes escolhas: Independentemente da renda, todos os 28\% das pessoas buscam uma alimentação mais saudável, portanto, a T'Entregas acabou se tornando um ponto forte porque mostrou a viabilidade do projeto e a possibilidade de sucesso. Ótima, rentabilidade.\par

\begin{commentA} \vspace{0.3cm} \noindent O que ele está comprando atualmente? \par \vspace{0.1cm} \end{commentA}

\begin{commentB}
Quadro: Hábitos de consumo.
\par \end{commentB}

\vspace*{0.3cm}
\begin{quadro}[!htbp]
\begin{footnotesize}
\captionsetup{textfont=bf, labelfont=bf, font=footnotesize, justification=centering}
	\caption {Hábitos de consumo.} \label{tab:habitos} 
	\vspace*{-0.3cm}
		\begin{tabular}{ |>{\raggedright\arraybackslash} m{9.5cm} | P{1.2cm}| P{1.5cm} | P{2cm} | } 
			\hline 
			%\rowcolor{gray}
			Hábitos de consumo                                                                     & Média & Desvio de erro & Coeficiente de variação \\ \hline %\toprule
			Costumo experimentar coisas novas e diferentes                                          & 3,54 & 1,113 & 31,46 \\ \hline %\midrule
			Costumo comprar novos produtos quando os vejo nas lojas, sites ou mídias de comunicação 
			& 3,10 & 1,148 & 37,04 \\ \hline %\midrule
			Gosto de sair e socializar com pessoas                                                  & 3,85 & 1,251 & 32,47 \\ \hline %\midrule
			Em geral, como refeições equilibradas e nutritivas                                      & 3,33 & 1,102 & 33,13 \\ \hline %\midrule
			Tomo cuidado com o que como                                                             & 3,85 & 0,952 & 24,71 \\ \hline %\bottomrule
		\end{tabular}
	\vspace*{-0.5cm}
	\par
	\bigskip
Fonte: Compilado pelo SPSS da pesquisa de campo
\end{footnotesize}
\end{quadro}
\par

Entre os entrevistados, 19,5\% compraram da Ifood, 16,5\% compraram da Uber Food, 12,38\% compraram da Rappi e as minorias étnicas (3,81\%) estavam procurando outros aplicativos.\par

Os dados mostrados na figura são concorrentes diretos da T'Entregas, pois essas empresas de Fortaleza têm concorrentes diretos e podem realizar pesquisas mais aprofundadas.\par

Porém, nas pesquisas realizadas com referência a essas aplicações, pode-se perceber que a maioria dos produtos possui cardápios pequenos de produtos saudáveis, mas como suas vantagens são os alimentos tradicionais, não há aprofundamento sobre o assunto.\par

Mesmo no caso de competidores diretos, devemos analisar cada competidor e sempre ir além do cardápio para oferecer cada vez mais alimentos fitness, sua qualidade, cuidado, limpeza, higiene, e sempre levar em consideração o bem-estar, saúde e qualidade As vidas dos potenciais os clientes fazem com que se tornem cada vez mais fiéis, atingindo assim os sucessivos objetivos da empresa.\par

\begin{commentA} \vspace{0.3cm} \noindent Por que ele está comprando? \par \vspace{0.1cm} \end{commentA}

\begin{commentB}
Gráfico: Frequência que consomem.
\par \end{commentB}


Conforme demonstrado na figura acima, a maioria dos entrevistados respondeu que se alimenta de alimentos saudáveis todos os dias, sendo 40,67\%, enquanto 28\% responderam que comem duas vezes por semana, enquanto apenas 7,33\% responderam que não comiam.\par

\vspace*{0.2cm}
\begin{figure}[!htbp]
\begin{footnotesize}
\captionsetup{textfont=bf, labelfont=bf, font=footnotesize, justification=centering}
	\caption {Ocasião de Consumo.} \label{tab:Ocasião de consumo} 
	\vspace*{-0.3cm}
	\pgfplotstableread[col sep=comma,header=true]{
		Type,N
		Lanche da manhã,17.19
		Lanche da tarde,16.45
		Almoço,14.23
		Jantar,29.76
		Ceia,11.09
		Eventos/reuniões,11.28
	}\data

		\begin{tikzpicture}     
			\begin{axis}[    
				width=0.88\textwidth,
				height=0.5\textwidth,
				bar width=0.6cm,
				xbar,                                 
				xtick={0,5,10,15,20,25,30,35},    
				xmin=0,
				xmax=35, 
				xmajorgrids,      
				%grid=major,
				nodes near coords, nodes near coords align={horizontal},
				symbolic y coords={Lanche da manhã,Lanche da tarde,Almoço,Jantar,Ceia,Eventos/reuniões},
				%ylabel={Type},
				%xlabel={Porcentagem},
				y label style={at={(-0.1,0.5)}},
				enlarge x limits={abs=0},
				%reverse legend,
				]
				\addplot [fill=RYB4] table [x=N, y=Type] {\data};
			\end{axis}
\end{tikzpicture}
\vspace*{-0.3cm}
\par
Fonte: Compilado pelo SPSS da pesquisa de campo
\end{footnotesize}
\end{figure}
\par

Eles levam uma vida indisciplinada, mas não descartam a possibilidade de tentar um bento fitness. Este pequeno público nos fornecerá pesquisas estratégicas mais aprofundadas para persuadi-los a usar nossas caixas de bento no futuro, e fazer com que nossos clientes (pesquisas pessoais mostram) nosso público-alvo potencial seja menor de dois anos.\par
 
Pesquisa analisada, entre 14 e 45 anos, jovens e adultos entre 14 e 45 anos têm o hábito de comer alimentos saudáveis quase todos os dias, e suas faixas salariais também são diferentes.

\begin{commentA} \vspace{0.3cm} \noindent * \par \vspace{0.1cm} \end{commentA}

\begin{commentA}
Quais fatores influenciam a compra?
\par \end{commentA}

\begin{commentB}
Gráfico : Entrevistados que comprariam
\par \end{commentB}

\vspace*{0.3cm}
\begin{figure}[!htbp]
\centering
\begin{footnotesize}
\captionsetup{textfont=bf, labelfont=bf, font=footnotesize, justification=centering}
	\caption {Local de solicitação.} \label{tab:Local de solicitação} 
	\vspace*{-0.3cm}
\begin{tikzpicture}
	
	\pie[text=legend, sum=auto, before number={}, after number=~\%]{30.74/Casa,25.29/Trabalho,22.57/Casa e Trabalho,21.40/Casa de amigos ou parentes}
	
\end{tikzpicture}
\vspace*{-0.3cm}
\par
Fonte: Compilado pelo SPSS da pesquisa de campo.
\end{footnotesize}
\end{figure}
\par

Os resultados obtidos mostram que 74\% dos entrevistados afirmaram que comprariam mercadorias em outras plataformas da região metropolitana de Fortaleza, e apenas 26\% dos entrevistados rejeitaram a ideia por afirmarem que não a desenvolveram. vegetais, feijão e alimentos integrais, como arroz e macarrão. O serviço prestado pela empresa deve ter uma diferença de 31\% na originalidade do sabor e outros fatores, para que, mesmo após a inauguração, passe a fazer parte da cultura epitaciana e contribua para a saúde pública geral. De acordo com os dados da Figura 5, esse negócio pode ser viável. Essas informações são muito relevantes para o projeto, pois deve-se destacar que uma parcela do público está interessada neste serviço diferenciado de lancheira fitness.

\begin{commentA} \vspace{0.3cm} \noindent * \par \vspace{0.1cm} \end{commentA}

\begin{commentA}
Quando, como e com que periodicidade é feita a compra?
\par \end{commentA}

\begin{commentA} \vspace{0.3cm} \noindent * \par \vspace{0.1cm} \end{commentA}

\begin{commentA}
Onde ele se encontra?
\par \end{commentA}

\begin{commentA} \vspace{0.3cm} \noindent * \par \vspace{0.1cm} \end{commentA}

\begin{commentA}
Como chegar até ele?
\par \end{commentA}






