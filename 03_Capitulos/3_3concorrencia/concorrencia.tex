\section{\textbf{Análise da concorrência}}
\label{sec: Análise da concorrência}

\begin{commentA}
Quadro sintético comparativo com: segmentos atendidos, proposta de valor, produtos/serviços oferecidos, relacionamento com clientes, comunicação com o mercado, precificação, canais de acesso para compra, reclamação/problemas, pontos fortes e fracos da concorrência.
\par \end{commentA}



\begin{commentA}
Uso da Curva Valor a partir do Mapa de Valor da ideia x Mapa Valor Concorrentes, outras informações da concorrência.
\par \end{commentA}



\begin{commentA}
Quem são os seus concorrentes?
\par \end{commentA}

Principais empresas de delivery no Brasil\par
\vspace{0.15cm}
iFood\par

Chegou ao Brasil em 2011 e ao Ceará em novembro de 2013. Atualmente operando na capital e outras 52 cidades. Entre novembro de 2018 e novembro de 2019, o número de restaurantes cooperativos aumentou de 52.000 para 131.300. E aumentou de 980.000 em novembro de 2018 para 6.5 milhões de pedidos em novembro de 2019.\par
\vspace{0.15cm}

Uber Eats\par
Chegou ao Brasil no final de 2016 e a Fortaleza em agosto de 2018. Atualmente, já se espalhou por mais de 100 cidades em todos os estados do Brasil. De 2018 a 2019, o número de restaurantes e parceiros de entrega triplicou.\par
\vspace{0.15cm}

Rappi\par
Ele chegou ao Brasil em julho de 2017 e Fortaleza em junho de 2018. Eles estão em 60 cidades do Brasil. Disponibilizandos produtos de restaurantes, supermercados e farmácias, além de outros serviços e produtos.\par
\vspace{0.15cm}

James Delivery\par
Começou a operar no ano 2019 e chegou ao Ceará em julho. Em relação ao início do ano, o número de pedidos no quarto trimestre de 2019 aumentou 15 vezes. Envie produtos de mercados, restaurantes, lojas de animais, farmácias e outros lugares.\par

\begin{commentA}
De que maneira o concorrente está organizado?
\par \end{commentA}

Praticidade são algumas das principais vantagens apontadas pelos usuários do aplicativo de entrega de pedidos de alimentos, e também são o foco do serviço. Uma variedade de descontos, entrega rápida, receba cupons de desconto.\par

\begin{commentA}
Ele pode tomar decisões mais rápidas que você?
\par \end{commentA}

Todo esse poder de fogo abriu o apetite por novos negócios: criando "dark kitchens" ou restaurantes virtuais que se concentram apenas na entrega. A aposta não é exclusiva do iFood: Rappi e Uber Eats também seguiram esse caminho para explorar a enorme quantidade de "big data" disponível.

O funcionamento é o seguinte: o serviço de entrega de alimentos detecta em seu banco de dados onde há uma grande demanda por um determinado produto. Eles escolhem parceiros para abrir restaurantes na região de forma exclusiva - geralmente o restaurante mais vendido da plataforma.

O aplicativo indica o melhor local para montar uma "cozinha virtual" - o mesmo espaço pode até reunir vários pratos diferentes (macarrão, pizza, sushi, hambúrguer, etc.), um prato que combina esses pratos juntos, Podendo ser usado no um aplicativo.

\begin{commentA}
Ele responde rapidamente a mudanças?
\par \end{commentA}

Pedir comida on-line por meio de um aplicativo pode tornar mais fácil para os consumidores, porque eles não precisam viajar longas distâncias ou fazer fila (e nem mesmo precisam estar ao telefone para servir a comida).\par 

Eles podem ir de qualquer lugar, navegar pelo menu, comparar restaurantes diferentes e receber sua escolha de alimentos sem esforço. Para começar a usar a plataforma e permitir que inúmeros consumidores a vejam, basta baixar o aplicativo e se cadastrar no site. Geralmente, esse é um processo muito simples e rápido, pois seu restaurante pode aparecer nos resultados de pesquisa do cliente em alguns dias.\par

O menu pode ser facilmente alterado ou atualizado por meio do aplicativo instantâneo. Fotos, preços de itens e novos pratos podem ser adicionados em poucos minutos. Assim como o serviço regular de restaurante, não há necessidade de reimprimir cardápios e brochuras toda vez que um prato é trocado.\par

O aplicativo mais popular tem uma grande base de usuários, e você pode encontrar facilmente seu restaurante com base na região em que atua e no nicho da indústria alimentícia. Assim, para além do apoio que um verdadeiro espaço de restaurante pode proporcionar, pode também aumentar o número de encomendas (e vendas!).\par

O uso desses aplicativos garante que sua empresa tenha maior reputação e valorize a marca do restaurante. Desta forma, você pode reduzir os custos com publicidade na Internet e nas redes sociais e, ao mesmo tempo, garantir o crescimento das vendas.\par

Como os consumidores podem acessar todo o cardápio (incluindo bebidas, doces e outros alimentos) e escolher seus próprios pratos de todo o mundo, eles podem fazer os pedidos com mais rapidez, sem ter que atender o telefone, e podem fazer os pedidos de forma mais completa. Outro fator muito importante é que por se tratar de um aplicativo, pode diminuir a chance de erros no pedido e diminuir o número de reclamações.\par


\begin{commentA}
Ele tem uma equipe gerencial eficiente?
\par \end{commentA}



\begin{commentA}
A concorrência é líder ou seguidora no mercado?
\par \end{commentA}



\begin{commentA}
Eles poderão vir a ser os seus concorrentes no futuro?
\par \end{commentA}




























