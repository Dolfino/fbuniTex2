\chapter{Conceito de Negócio}
\label{chapter:Conceito}


\begin{commentA} \vspace{0.3cm} \noindent Deve conter pequeno histórico da ideia.\par \vspace{0.1cm} \end{commentA}


Com 5 anos de experiência em uma grande empresa que desenvolve e licencia programas de computador customizáveis e atua em operações de mobilidade e marketplace de delivery, tenho o entendimento de melhorias de processos, para momento na área de operações em que os produtos são entregues sem danos, e as perdas causadas por mudanças ou reclamações dos clientes sobre os produtos entregues reduzida.\par

A maioria das pessoas procura um aplicativo que possa ajudá-lo a comprar comida de forma conveniente e entregar mercadorias onde quer que eu esteja sem um endereço físico, e que tenha uma interface intuitiva e fácil de usar, disponível para usar os recursos atuais processados pelo próprio aplicativo nos sistemas de smartphone mais comuns, e usar informações pessoais de forma seguras durante o registro, este é um dos principais problemas relacionados à segurança da informação.\par

Há dúvidas sobre o local a encontrar e o grau de proximidade com os usuários nos aplicativos por parte dos entregadores, que reduz a consciência dos usuários onde permanece, sobre a diversidade de opções de alimentos e produtos, o mesmo local de um produto é comum em outras categorias de alimentos, o que é outro problema.O prazo de entrega, que é um dos principais interesses dos usuários no aplicativo, desenvolver funções para que o usuário possa agendar o horário para receber o pedido sem demora.\par

O modelo de negócios da T'Entregas é uma espécie de plataforma multilateral que conecta consumidores que desejam receber alimentação com restaurantes que fornecem essa alimentação.\par
