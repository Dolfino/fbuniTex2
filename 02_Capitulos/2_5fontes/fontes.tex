\section{\textbf{Fontes de receitas}}
\label{sec: Fontes de receitas}

\begin{commentA} \vspace{0.3cm} \noindent  O que o cliente tem pago ultimamente para resolver o mesmo problema? \par \vspace{0.1cm} \end{commentA}


Aplicativo gratuito com opção Premium que possibilita ao cliente um acesso diferenciado com descontos nos produtos.\par

\begin{commentA} \vspace{0.3cm} \noindent O que o cliente valoriza e pelo qual está disposto a pagar? \par \vspace{0.1cm} \end{commentA}


O usuário da plataforma até aceita pagar mais por um atendimento self service, desde que ofereça uma experiência de compra tão maravilhosa, ao ponto de não deixar nenhuma dúvida sobre e o investimento na experiência de compra online será um dos valores entregue, já que este recurso é amplo e torna-se um atrativo para uma transação rápida e transparente de toda a operação.\par

\begin{commentA} \vspace{0.3cm} \noindent Como o cliente (que maneira) eles preferem pagar pelo valor gerado? \par \vspace{0.1cm} \end{commentA}


Pagamento adiantado, transação é online via app, com cobrança direta no cartão de crédito cadastrado ou com dinheiro quando o parceiro de entregas chegar no local.\par

\begin{commentA} \vspace{0.3cm} \noindent Qual é a parcela de contribuição (relevância) de cada fonte de receita para a receita total esperada? \par \vspace{0.1cm} \end{commentA}


A T’Entregas cobra uma comissão dos restaurantes de 25\% para obter os clientes, essa comissão é paga a cada pedido solicitado pela plataforma, também é cobra do parceiro de entrega 8\% pela a utilização do aplicativo que usar e está integrado a plataforma para entregar de comida, essa cobrança depende da taxa de entrega e do valor ganho.\par

\begin{commentA} \vspace{0.3cm} \noindent Alguns exemplos são:
Venda de Produtos, Preço por uso do produto, Preço por assinatura, Aluguel, Licença, Arbitragem (intermediação, agenciamento), Publicidade, Leilão, etc.\par \vspace{0.1cm} \end{commentA}

