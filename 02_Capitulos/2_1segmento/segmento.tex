\section{\textbf{Segmento de clientes}}
\label{sec: Segmento de clientes}


\begin{commentA} \vspace{0.3cm} \noindent Quem possui o problema a ser solucionado? \par \vspace{0.1cm} \end{commentA}


Conecta consumidores que desejam receber comida delivery e restaurantes que oferecem essa modalidade de entrega. Desta forma, através do T'Entregas você pode pedir comida de diferentes restaurantes com base em seus cardápios, preços, avaliações de outros clientes e tempo de entrega estimado. O T'Entregas trata-se de um aplicativo para smartphone que oferece um marketplace com centenas de restaurantes que trabalham com entrega de comida. A oferta do aplicativo é para pessoas maiores de 18 anos, que buscam comodidade e segurança nos momentos diversos e que desejam recebe uma refeição rápida, no local que se encontrar ou onde quiser.\par


\begin{commentA} \vspace{0.3cm} \noindent Para quem criamos valor? \par \vspace{0.1cm} \end{commentA}


Esse valor é para quem quer vivenciar pelo celular a entrega de comida que quer comer ou o produto que precisa ter pelo delivery e no mesmo ambiente pagar, pois a experiência se conclui quando o entregador a faz de maneira rápida, apresentando visualmente o produto bem embalado e na temperatura esperada.\par


\begin{commentA} \vspace{0.3cm} \noindent Quais seus hábitos? \par \vspace{0.1cm} \end{commentA}


O consumo é semanal neste mercado, os usuários fazem pedidos mas de duas vezes e outros fazem um pedido toda semana. os principais horários são: jantares nos dias de semana e almoço aos sábados e domingos.


\begin{commentA} \vspace{0.3cm} \noindent Quais são as características deste(s) segmento(s)? \par \vspace{0.1cm} \end{commentA}


O segmento é um modelo de plataforma multilateral, que atende a dois grupos de clientes diferentes que permitem aos usuários criar e consumir valor, criar valor enquanto interagem uns com os outros, enquanto conecta estabelecimentos com clientes que precisam de um entregador e serviços de entrega, torna-se um modelo que é altamente escalável, pois aproveita os efeitos de rede.\par


\begin{commentA} \vspace{0.3cm} \noindent Quem são os nossos potenciais clientes mais importantes? \par \vspace{0.1cm} \end{commentA}


Eles recebem publicidade e conectividade com o usuário final, o restaurante lista sua marca e cardápio na plataforma, o cliente faz o pedido usando o celular ou o computador e um dos entregadores da T'Entregas irá buscar a refeição no restaurante e entregá-la nas mãos do consumidor.\par

