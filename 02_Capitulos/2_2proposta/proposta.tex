\section{\textbf{Proposta de valor}}
\label{sec: Proposta de valor}

O que oferta os estabelecimentos comerciais parceiros a entregadores que buscam ganhos em dinheiro extra com flexibilidade de horário e a clientes que precisam recebe comida delivery dos melhores restaurantes da região ganhando tempo e recebendo a aonde estive da maneira mais rápida.\par


\begin{commentA} \vspace{0.3cm} \noindent Quais problemas dos clientes estamos resolvendo? Ou ajudando a resolver? \par \vspace{0.1cm} \end{commentA}



\begin{commentA} \vspace{0.3cm} \noindent Que valor criamos e entregamos para o cliente?\par \vspace{0.1cm} \end{commentA}



\begin{commentA} \vspace{0.3cm} \noindent Que necessidades dos clientes estamos satisfazendo?\par \vspace{0.1cm} \end{commentA}



\begin{commentA} \vspace{0.3cm} \noindent Que pacotes de produtos/serviços estamos oferecendo para cada segmento de Clientes? \par \vspace{0.1cm} \end{commentA}



\begin{commentA} \vspace{0.3cm} \noindent Exemplos:
B2C: Novidade, Performance, Customização, Funcionalidade, Design, Marca/Status, Preço, Redução de Custos, Redução de Riscos, Acessibilidade, Conveniência/Usabilidade, Geração de Receita, etc.
B2B: é importante pensar no que a sua oferta ajuda a empresa cliente a aumentar receitas, diminuir custos ou melhorar o serviço/produto.\par \vspace{0.1cm} \end{commentA}


%\begin{commentA}
%Sintetizar com um parágrafo:
%Nosso(s) (produtos/serviços) ___ ajuda(m) (segmento de clientes) ___ que deseja(m) (tarefas a realizar) _____ ao (reduzir , evitar etc.) ____ (dor do cliente) ____ e (aumentar, possibilitar) ____ (ganho do cliente), diferentemente de (proposta de valor da concorrência) ____ \par \end{commentA}
