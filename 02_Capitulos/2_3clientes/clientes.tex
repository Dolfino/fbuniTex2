\section{\textbf{Relacionamento com clientes}}
\label{sec: Relacionamento com clientes}


\begin{commentA} \vspace{0.3cm} \noindent Que tipo de relacionamento os clientes de cada segmento esperam? \par \vspace{0.1cm} \end{commentA}


Conecta consumidores que desejam receber entregas de alimentos e restaurantes que fazem essas entregas. Desta forma, com a T'Entregas, pode encomendar comida de diferentes restaurantes com base na categoria, preço, opiniões de outros clientes e prazo de entrega estimado, o público-alvo deste aplicativo são pessoas maiores de 18 anos. T'Entregas é um aplicativo para smartphone que pode fornecer centenas de restaurantes que oferecem comida para delivery.\par

O consumidor tem entrega diária de alimentos e as encomendas podem ser feitas em mais de 100 restaurantes, entre esses padarias e cafeterias. O aplicativo usa um sistema de previsão e geolocalização semelhante ao da Uber. Há sistema de pagamento eletrônico por meio do qual os usuários podem realizar pagamentos, apenas com o uso do celular.\par

Os entregadores podem ficar online o tempo que quiserem via T'Entregas, receber o pagamento sob demanda, sua receita pode dobrar para aqueles com entrega completa no periodo de 8 horas online. Há oportunidade para pessoas com nível de escolaridade básico, os entregadores podem cadastrar mais de um tipo de veículo, aumentando as possibilidades de entrega e liberdade de trabalho para cada um.\par


\begin{commentA} \vspace{0.3cm} \noindent Qual é o custo de cada um deles? (tipo de relacionamento) \par \vspace{0.1cm} \end{commentA}


O estabelecimento comercial reconhece e concorda que a T'Entregas prestará serviços e canais de suporte por meio do portal disponível no sistema, devendo o estabelecimento se comunicar com a T'Entregas por meio do portal e do e-mail nele fornecido. No portal, onde o estabelecimento aprova o pedido recebido para delivery pode-se consultar o número de encomendas que realiza, de que região provém a maior parte delas, quais os pratos que estão a vender cada vez menos, o tíquete médio de encomenda, entre outros.\par

Já o usuário em alguns casos, não se cobra nada dos clientes como forma de atraí-lo para então poder oferece-los ao outro segmento.\par

Os entregadores reconhecem e concordam que a T'Entregas não é uma empresa especializada em transporte ou operação logística, cabendo à T'Entregas tão somente disponibilizar uma plataforma tecnológica que possibilita a colaboração entre os que desempenham atividades relacionadas – assim, a atividade de entrega, bem como quaisquer perdas, prejuízos e/ou danos decorrentes ou relativas a tal atividade, são de responsabilidade exclusiva dos entregadores.\par


\begin{commentA} \vspace{0.3cm} \noindent O que pode se esperar em termos de interesse. \par \vspace{0.1cm} \end{commentA}


Estas são algumas das vantagens que pode o estabelecimento esperar em se tornar um parceiro da T'Entregas; aproveita o nosso conhecimento no mercado, para ganha visibilidade, atraindo novos clientes. Pode aumentar o seu faturamento em até 50\%, gerencia o seu negócio com muito mais facilidade com as nossas ferramentas e soluções. A T'Entregas também investe em marketing, o que atrai cada vez mais clientes para a plataforma e, consequentemente, muito mais vendas para os estabelecimentos parceiras.\par


\begin{commentA} \vspace{0.3cm} \noindent O que pode se esperar em termos de aquisição. \par \vspace{0.1cm} \end{commentA}


Os aplicativos são conhecidos por possuírem uma base de usuários muito grande, e, dependendo da região na qual o estabelecimento atue e do seu nicho dentro do ramo alimentício, o estabelecimento pode ser facilmente encontrado.\par


\begin{commentA} \vspace{0.3cm} \noindent O que pode se esperar em termos de retenção. \par \vspace{0.1cm} \end{commentA}


É possível medir o nível de satisfação após o recebimento do pedido, com isso saber quais são importantes para ele, o usuário pode avaliar os quesitos, como: tempo de entrega, experiência, e sugestões de melhorias escrita no comentário, assim é possível entender os fatores de decisão do cliente e dentro das características do estabelecimento,  o que o consumidor mais valoriza, para um trabalho de retenção e fidelização de clientes.\par


\begin{commentA} \vspace{0.3cm} \noindent O que pode se esperar em termos de up-selling (vendas complementares) \par \vspace{0.1cm} \end{commentA}


Um bom ambiente virtual é fundamental para que o upselling funcione, adicionar sugestões nas descrições de cada prato ou recomendações, precisam ser coerentes, intercalar itens caros com itens baratos. Saber os produtos mais vendidos (pelo delivery, nesse caso), pode ajudar a escolher como oferecer para aplicar o upsell.\par


\begin{commentA} \vspace{0.3cm} \noindent Como isso (esse tipo de relacionamento) está integrado ao Modelo de Negócio como um todo?
\par \vspace{0.1cm} \end{commentA}


No caso do modelo de negócio que é plataforma multilateral, é importante regras e recursos que criem confiança entre as partes, estabelecimentos e clientes.\par


\begin{commentA} \vspace{0.3cm} \noindent São exemplo de CRM:
As atividades de assistência pré e pós-venda com equipe dedicada, serviços automatizados, fóruns e comunidades de suporte, Co criação de conteúdo, etc. \par \vspace{0.1cm} \end{commentA}


