\section{\textbf{Estrutura de custos envolvidos na operação}}
\label{sec: Estrutura de custos envolvidos na operação}

\begin{commentA} \vspace{0.3cm} \noindent Quais são os custos mais importantes inerentes ao (nosso) modelo de negócio? \par \vspace{0.1cm} \end{commentA}


A estruturação de equipes de marketing é essencial para toda a empresa. A partir de profissionais qualificados e bem treinados é possível alcançar os objetivos da organização, aumentando seu desenvolvimento e crescimento. De forma geral, a equipe de marketing é a área responsável pela comunicação, distribuição, precificação e formatação dos produtos e serviços das empresas. Nesse contexto, é formada por profissionais qualificados que constroem estratégias e táticas para desenvolver a imagem da instituição e alavancar as vendas. A os parceiros: agências de performance; agências de publicidade; produtoras de áudio/vídeo; freelancers; dentre outros são acionados para garantir que os objetivos de marketing sejam alcançados.\par

\begin{commentA} \vspace{0.3cm} \noindent Quais recursos-chave são os mais caros? \par \vspace{0.1cm} \end{commentA}


O novo perfil de cliente está massivamente presente em ambientes virtuais, tornando as tecnologias web cruciais para o plano de estruturação do seu time de suporte (Central de relacionamento).\par

\begin{commentA} \vspace{0.3cm} \noindent Quais atividades-chave são as mais caras? \par \vspace{0.1cm} \end{commentA}


Sistemas de telefonia, por exemplo, devem ser integrados com um software de atendimento, como o CRM para otimizar e agilizar o processo de atendimento. Outro fator que também precisa ser considerado é permitir que a equipe atue pelos meios de comunicação usados pelo cliente (chat, e-mail, redes sociais etc.).\par

\begin{commentA} \vspace{0.3cm} \noindent Exemplos:
Custos fixos, custos variáveis, economias de escala, comissões, etc.
\par \vspace{0.1cm} \end{commentA}

