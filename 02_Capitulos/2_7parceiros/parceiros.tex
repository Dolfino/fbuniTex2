\section{\textbf{Parceiros}}
\label{sec: Parceiros}

\begin{commentA} \vspace{0.3cm} \noindent Quais são ou quais devem ser nossos principais parceiros-chave? \par \vspace{0.1cm} \end{commentA}


Os parceiros que manteremos comunicação e a manutenção de processos, são os estabelecimentos que disponibilizar o produto e o outro é o entregador, com que teremos uma visão diária do processo.\par

\begin{commentA} \vspace{0.3cm} \noindent E quais são os nossos principais fornecedores estratégicos? \par \vspace{0.1cm} \end{commentA}


Os estabelecimentos parceiros estão definidos na operação como estratégico, pois dependendo da localização se tornar um distribuidor principal.\par

\begin{commentA} \vspace{0.3cm} \noindent Quais recursos-chave estamos nos beneficiando (obtendo) deles? \par \vspace{0.1cm} \end{commentA}


A parceria com os estabelecimentos gera um percentual com a prestação do serviço de entrega.\par

\begin{commentA} \vspace{0.3cm} \noindent Quais atividades-chave são essenciais e quais eles produzem? \par \vspace{0.1cm} \end{commentA}


A essência por trás deste novo sistema de delivery é a entrega do produto logo após a conclusão, e forma rápida, com ao monitoramento até sua entrega.\par

\begin{commentA} \vspace{0.3cm} \noindent Exemplos:
Alianças estratégicas entre não-concorrentes, redes de cooperação entre concorrentes, joint ventures, parcerias de exclusividade, etc.
\par \vspace{0.1cm} \end{commentA}

