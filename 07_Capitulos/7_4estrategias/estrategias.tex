\section{\textbf{Estratégias de marketing}}
\label{sec: Estratégias de marketing}


\begin{commentA}

\par \end{commentA}

5.3.3. Estratégias promocionais 
Será por meio de estratégias promocionais que prestadores de serviços e consumidores finais serão convidados a utilizar a plataforma. Quanto mais prestadores de serviços oferecerem seus produtos no marketplace, mais consumidores irão aderir à plataforma, trazendo assim cada vez mais prestadores de serviços, tornando então num ciclo de ganha-ganha. Para isso acontecer, as estratégias promocionais deverão ser aplicadas de forma eficiente.  
As estratégias promocionais estabelecidas pela empresa se encontram no quadro abaixo: 

Figura 10 - Quadro Estratégias Promocionais 
Estratégia 	Descrição 
3 meses gratuitos 	Como forma de atrair prestadores de serviços à plataforma, os 3 primeiros meses de utilização serão gratuitos. 
Facebook 	Pelo fato de ser a rede social mais utilizada no mundo, é necessário que a empresa tenha uma página no Facebook. Por meio desta página, a empresa buscará aumentar sua presença criando reconhecimento no mercado, assim como também conquistar a fidelidade dos clientes. Será por meio do Facebook Ads (propaganda paga) que a empresa buscará gerar demanda 
e impulsionar as vendas. 
Google Adwords 	Com o objetivo de aumentar a visibilidade da marca, serão realizados anúncios por meio de links patrocinados na plataforma Google Adwords. 
